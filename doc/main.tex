\documentclass[
  french,
  a4paper,
]{scrartcl}

\usepackage[pages=some]{background}

\usepackage{amsmath,amssymb}
\usepackage[french]{babel}

\usepackage[T1]{fontenc}
\usepackage[utf8]{inputenc}
\usepackage{graphicx}
\usepackage{tabularx}
\usepackage{listings}

\usepackage{lmodern}

\usepackage[
  top=2cm, 
  bottom=3.5cm, 
  left=2cm, 
  right=2cm
  ]{geometry}
\usepackage[hidelinks]{hyperref}

\lstdefinestyle{mystyle}{
    backgroundcolor=\color{backcolour},   
    commentstyle=\color{codegreen},
    keywordstyle=\color{magenta},
    numberstyle=\tiny\color{codegray},
    stringstyle=\color{codepurple},
    basicstyle=\ttfamily\footnotesize,
    breakatwhitespace=false,         
    breaklines=true,                 
    captionpos=b,                    
    keepspaces=true,                 
    numbers=left,                    
    numbersep=5pt,                  
    showspaces=false,                
    showstringspaces=false,
    showtabs=false,                  
    tabsize=2
}
\lstset{style=mystyle}

\renewcommand{\arraystretch}{1.2}
\setlength {\parindent}{0em}
\setlength {\parskip}{1em}

\usepackage{scrlayer-scrpage}

\setlength{\headsep}{0.5cm}
\setlength{\headheight}{2cm}

\makeatletter
\renewcommand*\maketitle{
  \begin{center}
    \sffamily
  
    \vspace*{10pt}
    \Large 3259.1 Paradigmes de programmation avancés II $\cdot$ 2023-2024 $\cdot$ ISC3il-a

    \line(1,0){170mm}

    \huge \textbf{Laboratoire 2 : Monitoring de la concurrence avec le pattern producteurs/consommateurs en Java}    

    \line(1,0){170mm}
  \end{center}

    \vspace{10pt}
    
    \sffamily
    \Large Nima Dekhli \hfill \today
    
    \vspace{8pt}
}

\makeatother

% setup header on title page
\clearpairofpagestyles

\newpairofpagestyles{firstpage}{
   \ihead{
    \includegraphics*[height=1.5cm]{img/he-arc.png}
  }
  \ohead{
       \includegraphics*[height=1.0cm]{img/hes-so.png}
  } 
}


\begin{document}
\maketitle
\rmfamily
\normalsize
\tableofcontents
\thispagestyle{firstpage}
\newpage

\section{Introduction}

Dans le cadre d'un laboratoire du cours de paradigmes de programmation avancés II, 
il nous a été demandé de réaliser un programme de monitoring de la concurrence 
en utilisant le pattern producteurs/consommateurs. En particulier, 
il nous a été demandé de mesurer les performances, telles que débit 
et la latence. 


\section{Implémentation}

\subsection{Buffer}

Afin d'implémenter le pattern de producteurs/consommateurs, on utilise un 
Buffer circulaire, implémenté dans la classe \lstinline{ch.hearc.nde.buffer.CircularBuffer}. 

% TODO: 

\subsection{Producteurs et consommateurs}

% TODO:

\subsection{Modification en temps réel des paramètres}

Le programme permet une modification en temps réel du nombre de consommateurs, de 
producteur, de la taille du buffer et de la taille d'échantillonnage. 

Lorsqu'un de ces paramètres est modifié, on réinitialise le programme. On commence
par arrêter tous les threads producteurs et consommateurs, on vide le buffer et on
réinitialise les statistiques. Ensuite, on recrée les threads producteurs et consommateurs
avec les nouveaux paramètres.

Ceci permet d'éviter des problèmes de concurrence lors de la modification des paramètres mais 
cela implique un certain délai lors de la modification des paramètres et des 
statistiques qui ne sont pas représentatives pendant ce laps de temps.

\subsection{Mesures}

\subsubsection*{Méthode}

Afin de mesurer le débit (quantité de données traitées par unité de temps) et la 
latence (délai entre la production et la consommation de la donnée), une classe 
de gestion des statistiques est contenue dans le gestionnaire de buffer. 
A chaque lecture et écriture, 
un appel est effectué sur le gestionnaire de statistiques pour le mettre à jour.

Afin de mesure le \textbf{débit} (quantité de données traitées par unité de temps) et la 
\textbf{latence} (délai entre le début de la production et la fin de la consommation), une 
classe de gestion des statistiques a été créée. Elle permet de notifier un début de production, 
une fin de consommation et la récupération des statistiques. 

Afin de calculer la latence sur une même donnée, on fourni un hash unique à chaque pour 
identifier la donnée. En l'occurrence, on utilise la méthode \lstinline{String.hashCode()}
sur le message, qui renvoie un nombre entier déterminé par le contenu de la chaîne. Comme 
chaque producteur génère un message unique à chaque itération (composé de 
l'indice d'itération et de l'identifiant du producteur), on peut utiliser ce hash 
comme identifiant. 

Les valeurs statistiques (latence et débit) sont moyennés sur la taille d'échantillonnage. Cela signifie 
qu'on effectue une moyenne sur les $N$ derniers éléments qui ont été traités (produits et consommés). 
Cela signifie également que si la pipeline est bloquée (par exemple parce que la taille du buffer est de 0), 
les mesures ne seront pas immédiatement influencées. Une fois la pipeline débloquée, on pourra 
éventuellement voir une augmentation de la latence, qui dépendra du temps total de bloquage, de la 
taille d'échantillonnage et de la taille du buffer. 

\subsubsection*{Classe utilitaire de gestion des statistiques}

La gestion des statistiques est effectuée par la classe \lstinline{ch.hearc.nde.manager.StatisticsManager}. 
Avant de commencer la production, la méthode \lstinline{StatisticsManager.produce(String msg)} est appelée. 
Cela permet de notifier le début de la production d'un élément. Après avoir consommé l'élément, 
la méthode \lstinline{StatisticsManager.consume(String msg)} est appelée. Cela permet de notifier la fin
de la consommation de l'élément. 

\subsubsection*{Latence}

On définit la latence comme le temps écoulé entre le début de la production et la fin de la consommation. 

A chaque appel de \lstinline{StatisticsManager.produce(String msg)}, on enregistre le timestamp actuel dans une Map 
qui associe le hash du message à un timestamp. Au moment de l'appel à \lstinline{StatisticsManager.consume(String msg)},
on récupère le timestamp associé au hash et on calcule la différence avec le timestamp actuel. On ajoute 
cette différence à une liste de latences. Si la liste excède la taille d'échantillonnage, on retire le premier
élément de la liste (le plus ancien). 
Pour calculer la latence moyenne, on retourne la moyenne de tous les éléments de cette liste.

\subsubsection*{Débit}

On définit le débit comme le nombre d'éléments entièrement traités par unité de temps. Cela correspond
ainsi en réalité au débit de consommation, puisque le débit de production est limité par ce dernier.

A chaque appel de \lstinline{StatisticsManager.consume(String msg)}, on ajoute le timestamp actuel à une liste de
timestamps. Si la liste excède la taille d'échantillonnage, on retire le premier élément de la liste (le plus ancien). 
Pour calculer le débit, on divise le nombre d'éléments dans la liste par la différence entre le timestamp le plus récent et 
le plus ancien. 

\subsection{MBeans et JConsole}

Afin de simplifier les tests et de permettre une certaine interactivité, on utilise 
les MBeans qui peuvent ensuite être lues et modifiées avec la JConsole. 

Les valeurs qui peuvent être modifiées en temps réel sont les suivantes :

\begin{itemize}
  \item Le nombre de consommateurs 
  \item Le nombre de producteurs 
  \item La taille du buffer
  \item La taille d'échantillonnage
\end{itemize}

Les valeurs qui peuvent uniquement être lues sont les suivantes : 

\begin{itemize}
  \item Le débit moyen (en nombre d'éléments traités par seconde, moyenne selon la taille d'échantillonnage)
  \item La latence moyenne (en millisecondes, moyenne selon la taille d'échantillonnage)
  \item Le nombre d'éléments actuellement dans le buffer (éléments produits mais pas encore consommées)
\end{itemize}

Pour chaque élément pouvant être modifié, on ajoute un \textit{getter} et un \textit{setter} dans l'interface
\lstinline{ch.hearc.nde.manager.BufferManagerMBean} et pour chaque élément pouvant être lu, on ajoute un \textit{getter}. 
La classe \lstinline{ch.hearc.nde.manager.BufferManager} implémente cette interface et est enregistrée en tant que MBean
au début de l'exécution du programme.


\section{Protocoles de tests et résultats attendus}

Dans cette section sont détaillées les différentes mesures que nous effectuerons sur le programme, 
ainsi que les résultats attendus. On souhaite en particulier observer l'impact de la taille du buffer, 
du nombre de producteurs et du nombre de consommateurs sur le débit et la latence. 

On divisera alors le protocole en trois parties : Lorsque le nombre de producteurs est égal 
au nombre de consommateurs, lorsque le nombre de producteurs est supérieur au nombre de consommateurs
et lorsque le nombre de producteurs est inférieur au nombre de consommateurs. Dans chacun des cas, 
on évaluera également l'impact de la taille du buffer. 

\subsection{Nombre de producteurs égal au nombre de consommateurs}

\subsubsection*{Valeurs de test}

\begin{table}[h]
  \centering
  \begin{tabular}{l|l}
    \textbf{Attribut}  & \textbf{Valeur} \\
    \hline
    Nombre de producteurs & 100 \\
    Nombre de consommateurs & 100 \\
    Taille du buffer & 1000 \\
  \end{tabular}
\end{table}

\subsubsection*{Résultats attendus}

%TODO: change the value xxx
Comme les producteurs et les consommateurs ont chacun un délai constant et connu (xxx ms) pour 
produire et consommer un élément, il ne devrait pas y avoir de goulot d'étranglement et le débit 
devrait être maximal et la latence minimale (pour le nombre de producteurs et consommateurs choisis).

\subsubsection*{Débit théorique}

$$
d_{th} = \frac{\text{Nombre de consommateurs}}{\text{Temps de consommation}} = \frac{100}{0.5} = 200 \text{ éléments/s}
$$

\subsubsection*{Latence théorique}

$$
l_{th} = \text{Temps de production} + \text{Temps de consommation} = 0.5 + 0.5 = 1 \text{ s}
$$

\subsection{Nombre de producteurs supérieur au nombre de consommateurs}

\subsubsection*{Valeurs de test}

\begin{table}[h]
  \centering
  \begin{tabular}{l|l}
    \textbf{Attribut}  & \textbf{Valeur} \\
    \hline
    Nombre de producteurs & 1000 \\
    Nombre de consommateurs & 100 \\
    Taille du buffer & 1000 \\
  \end{tabular}
\end{table}

\subsubsection*{Résultats attendus}

Dans ce cas, le débit devrait entièrement être limité par les consommateurs. Ajouter plus de 
producteurs ne le fera donc pas augmenter. Toutefois, la latence sera influencée par le grand 
nombre de producteurs. En effet, ces derniers devront attendre que le buffer soit libéré pour
pouvoir y écrire. Comme ils sont 10 fois plus nombreux que les consommateurs, on s'attend 
à ce que la latence soit 10 fois plus grande que dans un cas équilibré.

\subsubsection*{Débit théorique}

$$
d_{th} = \frac{\text{Nombre de consommateurs}}{\text{Temps de consommation}} = \frac{100}{0.5} = 200 \text{ éléments/s}
$$

\subsubsection*{Latence théorique}

$$
l_{th} = \frac{\text{Nombre de producteurs}}{\text{Nombre de consommateurs}} \cdot (\text{Tps de production} + \text{Tps de consommation}) = \frac{1000}{100} \cdot (0.5 + 0.5) = 10 \text{ s}
$$

\subsection{Nombre de producteurs inférieur au nombre de consommateurs}

\subsubsection*{Valeurs de test}

\begin{table}[h]
  \centering
  \begin{tabular}{l|l}
    \textbf{Attribut}  & \textbf{Valeur} \\
    \hline
    Nombre de producteurs & 100 \\
    Nombre de consommateurs & 1000 \\
    Taille du buffer & 1000 \\
  \end{tabular}
\end{table}

\subsubsection*{Résultats attendus}

Dans ce cas, les producteurs travailleront à plein régime, mais les consommateurs devront 
attendre qu'une donnée soit disponible dans le buffer pour la consommer. Le débit sera donc 
limité par le nombre de producteurs. On pourrait penser que la latence sera également augmentée 
car les consommateurs devront attendre plus longtemps pour obtenir une donnée. Toutefois,
comme la latence est calculée sur le temps qu'une \textbf{donnée} met à être produite et consommée,
elle n'augmentera pas. En effet, une fois produite, elle est immédiatement consommée par l'un 
des nombreux consommateurs disponibles et en attente. 

\subsubsection*{Débit théorique}

$$
d_{th} = \frac{\text{Nombre de producteurs}}{\text{Temps de production}} = \frac{100}{0.5} = 200 \text{ éléments/s}
$$

\subsubsection*{Latence théorique}

$$
l_{th} = \text{Temps de production} + \text{Temps de consommation} = 0.5 + 0.5 = 1 \text{ s}
$$

\section{Résultats}

Dans cette section sont présentés les résultats des tests effectués. 

\section{Analyse}

Dans cette section sont analysés et discutés les résultats obtenus. En particulier, 
on compare les résultats obtenus avec les résultats attendus.

\section{Conclusion}
 
 
\end{document}